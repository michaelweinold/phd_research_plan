\documentclass{article}
% formatting
\usepackage{auxiliary/style_files/arxiv}
\usepackage[utf8]{inputenc} % allow utf-8 input
\usepackage[T1]{fontenc}    % use 8-bit T1 fonts
% cross-referencing
\usepackage{hyperref}
% tables
\usepackage{booktabs}
\usepackage{tabularx}
\usepackage{multirow}
\usepackage{caption} 
\usepackage{float} % for forcing placement
\captionsetup[table]{skip=10pt}
% fonts
\usepackage{amsfonts}
\usepackage{microtype}
% color
\usepackage{color}
\definecolor{verbgray}{gray}{0.9}
% figures
\usepackage{graphicx}
% enumeration
\usepackage{enumitem}
% embedd pdf files
\usepackage{pdfpages}

% code
\usepackage{listings} % https://tex.stackexchange.com/a/53951
\lstnewenvironment{code_search}{
    \lstset{
        backgroundcolor=\color{verbgray},
        frame=single,
        basicstyle=\ttfamily,
        %columns=fullflexible,
        xleftmargin=0cm,
    }
}{}
\setcounter{secnumdepth}{5}

\usepackage{footnote}
\makesavenoteenv{tabular}

\title{Prospective Hybrid Life-Cycle Assessment for Sustainable Aviation}


\author{
    \href{https://orcid.org/0000-0003-4859-2650}
    {\includegraphics[scale=0.06]{auxiliary/figures/orcid.pdf}
    \hspace{1mm}
    Michael Weinold} \\
	Paul Scherrer Institut\\
	Laboratory for Energy Systems Analysis\\
	Group for Technology Assessment\\
	OSHA/D22\\
    5232 Villigen PSI \\
    Switzerland \\
	\texttt{michael.weinold@psi.ch} \\
}

\date{19. October 2022}

\renewcommand{\headeright}{Doctoral Research Plan}
\renewcommand{\undertitle}{Doctoral Research Plan}

\hypersetup{
    pdftitle={Doctoral Research Plan},
    pdfauthor={Michael Weinold},
}

\begin{document}

\includepdf[pages=-]{attachments/research_plan_flat.pdf}

\maketitle

\begin{abstract}

    In light of international efforts to reduce carbon emissions and other environmental burdens, life-cycle assessment has emerged as a powerful tool for decision makers. Following international standardization efforts, the method is now widely used in academia and industry. However, depending on the sector or product under investigation, results of different life-cycle assessments still underestimate environmental burdens by up to 100\%. This is due to the natural limitation in the level of detail provided by process-based inventory data. This can greatly diminish confidence in results and negatively impact decisions taken on the basis of life-cycle assessments. While the issue has been at the forefront of discussions about the limitations of life-cycle assessments, it has remained unsolved at the practical level. This shortcoming can be overcome by hybridizing with environmentally extended input-output data, which covers the global economy in its entirety. However, adoption of this important improvement is hampered by a lack of understanding of the mathematical nuances of the proposed methods, as well as any work comparing them against important quality metrics, such as truncation errors. A lack of documentation in existing studies and the lack of any scientific software capable of hybridization further impede adoption.
    
    The proposed research project will develop a modular solution for hybridizing life-cycle inventory. This will provide the basis for the first systematic analysis of different proposed hybridization methods. Not only would this constitute a seminal contribution to the toolkit available to practitioners, it would address longstanding criticism of life-cycle methodology, greatly improve confidence in assessment results and improve the reproducibility of future studies. To optimize the access of life-cycle practitioners and researchers to the toolkit, it will be written in \texttt{Python} and integrated into the \texttt{Brightway} package. Further integration into the WISER InnoSuisse Flagship web service will enable industry stakeholders to take advantage of the aforementioned improvements immediately after their release. The proposed research project further builds on strong research foundations at the Technology Assessment Group of Paul Scherrer Institute.

    To demonstrate the capabilities of the toolkit, a selection of competing sustainable aviation technologies sustainable will be assessed. This will provide valuable insight to policy makers at a crucial moment when market forces have not yet eliminated solutions which are either economically unfeasible or environmentally objectionable. 
	
\end{abstract}

\section{Introduction}
\label{sec:introduction}
    
    As human economical activity has increased exponentially over the past century, anthropogenic pressure on the natural ecosystem has grown simultaneously \cite{steffen_anthropocene_2007}. This is illustrated by a progressive deterioration of the of Planetary Boundary indicators, defined as a set of environmental control variables under which human societies has historically flourished \cite{rockstrom_planetary_2009}\cite{steffen_planetary_2015}. In order to combat this decline effectively and avoid non-linear, abrupt environmental change, the Intergovernmental Panel on Climate Change has reiterated the importance of 
    \textit{"enhancing knowledge on risks, impacts, and their consequences"} \cite{rama_climate_2022} as part of evidence-based responses to current pressure on the natural ecosystem. In this context, life-cycle assessment has emerged as the primary method for assessing the environmental impacts associated with the complete life cycle of products, processes or services \cite{hauschild_life_2018}. Developed in the 1960s and following international standardization efforts, the methodology today is defined by ISO14044 \cite{noauthor_iso_2006}, developed by the International Organization for Standardization and the International Reference Life Cycle Data System, developed by the European Commission \cite{european_commission_joint_research_centre_institute_for_environment_and_sustainability_ildc_2010}, among others. A versatile method, it can be used to calculate the carbon footprint of products and processes \cite{finkbeiner_carbon_2016} or assess the environmental impact of entire organizations \cite{finkbeiner_life_2016}. Besides the obvious application in the direct comparison of different technologies, life-cycle assessment today provides a vital evidence-based environmental decision making tool for industry leaders and policy makers alike \cite{tillman_significance_2000}\cite{seidel_application_2016}\cite{dong_environmental_2018}. What is more, life-cycle assessment will be at the core of mandatory European Sustainability Reporting Standards, currently under development \cite{noauthor_public_2022}. It is thus that not only life-cycle practitioners and scientists have a vested interest in the method being as accurate as possible, but also industry leaders and political decision makers. 
    
    The widespread adoption of life-cycle assessment in academic publications and industry reporting non-withstanding, several limitations have opened the method to criticism \cite{reap_survey_2008}\cite{reap_survey_2008-1}\cite{finnveden_limitations_2000}\cite{plevin_using_2014}. The most significant drawback currently inherent to classical life-cycle assessment is referred to as \textit{truncation error} \cite{crawford_hybrid_2018}. For instance, in order to obtain the environmental impact of a product, a life-cycle inventory database is used to obtain information on all processes related to its manufacturing. Since the task of gathering information on all processes of the modern economy is daunting even at the most basic level of detail, such process-based databases are naturally limited in their scope. Hence, purely process-based life-cycle assessment is prone to underestimating (truncating) the full environmental impact. In fact, several meta-studies estimated the truncation error in specific case studies that used purely process-based life-cycle inventory and found an underestimation of up to 100\%, as shown in figure \ref{fig:underestimation}. This is closely related to the issue of choosing system boundaries in life-cycle assessment \cite{teh_hybrid_2017}. On the other hand, environmentally extended input-output analysis has attempted to circumvent this issue by using instead an inventory based on trade tables describing the monetary flows between different sectors of the economy. While this data is by definition complete, it does not provide the same level of process specific details and excludes important parts of the life cycle of products, such as end-of-life management.
    
    \begin{figure}[h!]
    	\centering
    	\includegraphics[width=\textwidth]{figures/underestimation_excel.png}
    	\caption{Compilation of the ranges in truncation errors for different process-based life-cycle assessment case-studies. Each meta-study on the abscissa gives a range for the truncation error of various purely process-based life-cycle assessment case-studies on the ordinate. The magnitude of the truncation is dependent on the parameters of the assessment, on the level of detail in the input data and the specific process or industry under investigation. Note that some studies underestimate emissions by as much as 100\%. Source: Compiled from data provided in reviews \cite{rebitzer_input-output_2002}\cite{lenzen_errors_2000}\cite{agez_hybridization_2020}\cite{agez_correcting_2022}\cite{agez_lifting_2019}.}
    	\label{fig:underestimation}
    \end{figure}
    
    Ideally, the methods could be combined to take advantage of the depth of process-based inventories and the breadth of input-output inventories. Such a hybrid life-cycle assessment requires a number of purely technical innovations related to the unification of nomenclature between databases, the handling of matrices and high levels of manual intervention. However, the most significant problem stems from the issue of double-counting emissions \cite{agez_hybridization_2020}. If, for instance, all processes related to steel production and use are covered in the process inventory, any additional inputs from an environmentally extended input-output database must not contain steel in order to avoid "double counting" of emissions.
    
\section{Current State of Research}
\label{sec:current_state_of_research}

    %\subsection{Implementation of Greenhouse Gas Assessment Frameworks in Life-Cycle Assessment Software}
    %    
    %    TEXT TO BE ADDED HERE
    
    \subsection{Life-Cycle Assessment}
    
        The current state-of-the-art in conventional process-based life-cycle assessment and inventory analysis is best documented in the book series \textit{Sustainable Production, Life Cycle Engineering and Management} \cite{teuteberg_progress_2019}\cite{albrecht_progress_2021}. A similar treatment of Leontief's input-output analysis is provided in a recent comprehensive book \cite{miller_input-output_2022}. The review of either is out of scope for this report.

    \subsection{Hybrid Life-Cycle Assessment}
        
        As detailed in the introduction, the limitations inherent to both analysis based on process inventories and analysis based on environmentally extended input-output tables motivated the initial introduction of a hybrid methodology for energy usage analysis by Bullard et al. in 1978 \cite{bullard_net_1978}. The term "hybrid life-cycle assessment" first saw use after 2000, for instance in a review by Lenzen et al. \cite{lenzen_errors_2000}. Since then, a number of case studies using hybrid methodology has been published, as illustrated in table \ref{tab_studies}.
        
        However, the methods used would only be formalized two decades later. The path-exchange method, proposed in 1997 by Treloar \cite{treloar_extracting_1997} and formalized in 2009 by Lenzen et al. \cite{lenzen_path_2009}. The matrix augmentation method, introduced in 1999 by Joshi \cite{joshi_product_1999}. The integrated method, introduced in 2000 by Suh and Hupped \cite{suh_gearing_2000}. The tiered method proposed in 1978 by Bullard et al. \cite{bullard_net_1978} and formalized by Heijungs et al. in 2002 \cite{heijungs_computational_2002}, which was further improved to avoid double counting by Agez et al. between 2019 and 2022 \cite{agez_lifting_2019}\cite{agez_hybridization_2020}\cite{agez_correcting_2022}.

        Three reviews specific to hybrid methods were identified. The most comprehensive treatment is given by Crawford in 2018 \cite{crawford_hybrid_2018} building on earlier work by Islam et al. \cite{islam_review_2016} and Nakamura at al. \cite{nakamura_inputoutput_2016}. Crawford et al. detail the concerning lack of even a unified nomenclature to designate different mathematical approaches to hybridization. In an attempt to provide such a standard, they collected the different nomenclature used by authors to describe their own methodology, illustrating the lack of agreement and poor understanding of the underlying mathematics of these methods. 
        
        \begin{figure}[h!]
        	\centering
        	\includegraphics[width=\textwidth]{figures/nomenclature.png}
        	\caption{Histogram showing the diversity in nomenclature used by authors of different life-cycle assessment case studies to self-define the methods used to hybridize life-cycle inventory. Shown are peer-reviewed publications between 2010 and 2015. The definition proposed by Crawford et al. is used to categorize different studies (Tiered, Matrix, Integrated, Path Exchange). Source: based on data collected by Crawford et al. \cite{crawford_hybrid_2018}. Note that table \ref{tab_studies} includes more recent studies studies beyond 2015.}
        	\label{fig:nomenclature}
        \end{figure}
        
        The literature review confirms findings by Crawford et al. on the three main aspects slowing down the uptake of hybrid methods in life-cycle assessment (adapted from \cite{crawford_hybrid_2018}):
        
        \begin{enumerate}[nosep]
            \item a lack of clarity in the description of the methods used in studies, making reproduction of methods difficult
            \item a lack of understanding of the benefits of using hybrid data over conventional process or input-output data
            \item a lack of automated tools that would allow these methods to be easily used by non-hybrid LCA specialists
        \end{enumerate}

        \begin{table}[H]
            \centering
            \begin{tabularx}{\textwidth}{| X | X | X | X |}
                \hline
                    \textbf{Study} & \textbf{Sector} & \textbf{Methodology} & \textbf{Indicators} \\
                \hline
                    Li et al., 2019 \cite{li_economic_2019} & Resource Extraction & IO + PLCI & CO2, SO2, NOx, soot \\
                \hline
                    Li et al., 2020 \cite{li_life_2020} & Agriculture & EIO + PLCI, MC-based sensitivity analysis & 10 from TRACI v2.1 \\ % . The results show that environmental impact from EIO-based inventory contributes a meaningful fraction of the impact for ozone depletion (67%), respiratory effects (42%), fossil fuel depletion (38%), and smog (28%) (as opposed to process-based inventory)
                \hline
                    Krishnan et al., 2004 \cite{krishnan_using_2004} & Semiconductors & bespoke IO + PLCA & electricity \\
                \hline
                    Wang et al., 2014 \cite{wang_hybrid_2014} & Semiconductors & EIO + LCI (stochiometric approach for chemicals) & human toxicity (cancer), water eutrophication, GWP, etc. \\
                \hline
                    Stokes et al., 2008 \cite{stokes_energy_2009} & Water Supply & EIOLCA tool \footnote{\url{http://www.eiolca.net/}} & CO2, SO2, NOx, soot, etc. \\
                \hline
                    Murray et al., 2008 \cite{murray_hybrid_2008} & Wastewater Treatment & EIOLCA tool \footnote{\url{http://www.eiolca.net/}} & CO2, SO2, NOx, soot, etc. \\
                \hline
                    Teh et al., 2017 \cite{teh_hybrid_2017} & Raw Materials Production & IELab tool \footnote{\url{https://ielab.info/}} & CO2 \\
                \hline
                    Facanha et al., 2006 \cite{horvath_environmental_2006} & Transportation (Land) & EIO + PLCI & CO2, NOx, PM10, CO \\
                \hline
                    Ewing et al., 2011 \cite{ewing_insights_2011} & Transportation (Sea) & EIOLCA tool \footnote{\url{http://www.eiolca.net/}} & CO2eq \\
                \hline
                    Martinez-Corona et al., 2016 \cite{martinez-corona_hybrid_2017} & Electricity (Geotherm.) & THEMIS tool \cite{gibon_methodology_2015} & GWP, etc. \\
                \hline
                    Vasan et al., 2014 \cite{vasan_carbon_2014} & Electronics Manufacturing & EIO + PLCI & CO2eq \\
                \hline
                    Arvesen et al., 2014 \cite{arvesen_life_2014} & Electricity (Wind) & EIO + PLCI & multiple \\
                \hline
                    Yao et al., 2014, \cite{yao_hybrid_2014} & PV Manuf. & EIO + PLCI & multiple \\
                \hline
                    Zhao et al., 2019 \cite{zhao_comparative_2019} & Li-Ion Battery Manuf. & detailed (follow-up) & multiple \\
                \hline
                    Wu et al., 2019 \cite{wu_assessing_2019} & Transportation (Land) & THEMIS tool \cite{gibon_methodology_2015} & CO2eq \\
                \hline
                    Ramaswami et al., 2008 \cite{ramaswami_demand-centered_2008} & Urban Emissions & generic & CO2eq \\
                \hline
                    Chang et al., 2014 \cite{chang_shale--well_2014} & Resource Extraction & generic? & CO2eq, SO2, NOx, CH4 \\
                \hline
            \end{tabularx}
            \caption{Selection of environmental impact studies using a hybrid life-cycle assessment methodology. Studies excluded are those older than 5 years with less than 5 citations and those describing sectors deemed too specific (e.g. \textit{"Comparative life cycle assessment of drinking straws in Brazil" \cite{zanghelini_comparative_2020}}). Note that no two studies used the same exact methodology and all studies used either fully manual or semi-manual methods to hybridize inventory. For a more comprehensive list, compare also \cite{crawford_hybrid_2018}. Abbreviations: IO = Input Output Table/Database, PLCI = Process-Based Life-Cycle Inventory Database, MC = Monte-Carlo Technique}
            \label{tab_studies}
        \end{table}
        
        In this context, the most significant contribution to the practical application of hybridization methodology identified since the respective formalizations is work by Agez et al. on introducing a heuristics-based semi-automated approach to avoiding double counting in tiered hybrid approaches \cite{agez_lifting_2019}\cite{agez_hybridization_2020}\cite{agez_correcting_2022}. The associated scientific software package, however, is poorly documented and includes no tests. Its output can thus not be easily verified.

    %\subsection{Prospective Hybrid Life-Cycle Assessment of Sustainable Aviation}
    
    %    No publication was identified using hybrid life-cycle assessment for the prospective assessment of sustainable aviation.
        
    %    include the work by sacchi et al.
    %    include the work by patt et al.
        
    
\section{Aviation}

https://www.aiazero.org/blog/climate-neutrality-not-just-carbon-neutrality-blog/
    
\section{Research Questions and Proposed Research Project}
    
    The importance of improving the accuracy of life-cycle assessment by adapting existing methodology to use hybrid inventories has been introduced in section \ref{sec:introduction}. The associated scientific groundwork has been laid out in section \ref{sec:current_state_of_research}. However, as is evident from the associated literature review, there is currently no consensus among the scientific community on the applicability of the different methods proposed. This is because of a lack of understanding of the downstream effects of different hybridization methods and the associated choices made on inventory matrices. There is further a lack of tools to perform automated hybrid life-cycle assessment, which in turn makes any meta-analysis of methods exceedingly difficult. A lack of universal nomenclature provides a further obstacles to progress in the field. 
    
    It is in this context that the present research project aims to advance the field by first building a toolkit for modular life-cycle inventory hybridization. Using this toolkit, for the first time, different methods for hybridization will be implemented comparatively and quality metrics, such as the truncation error, systematically quantified. This will allow for a seminal meta-study of the different hybridization methods and will enable the most accurate life-cycle assessment to-date.
    
    To demonstrate this, the air transport sector will be studied, providing a much more accurate prediction of sustainable aviation technologies than currently available. Sustainable aviation in particular would benefit from proper treatment by hybrid life-cycle assessment. Even recent publications in \textit{Nature Climate Change} completely \textit{"(...) exclude life-cycle emissions of zero-carbon fuels assuming that in a world committed to climate neutrality and aligned with the Paris Agreement, these are negligible and already neutralized as part of industrial and energy mitigation efforts."} \cite{brazzola_definitions_2022}, thereby disregarding the significant impact of life-cycle emissions associated with current and proposed fuels and technologies \cite{prussi_corsia_2021}.
    
    In conclusion, the research project proposed in the present document and its resulting publications and scientific software will constitute a significant contribution to one of the most significant limitations in a key method in evidence-based environmental decision making. Improvements in the accuracy of the environmental impact calculations will increase confidence in the method itself while the technical implementation will enable, for the first time, practitioners and researchers to access the full potential of hybrid life-cycle assessment.

    %Carbon emissions estimations for individual flights have been available from major flight search providers, including \textit{Google Flights} \cite{holden_google_2021}, \textit{Kayak} \cite{noauthor_kayak_2021} and \textit{SkyScanner} \cite{crosthwaite_how_2021} from 2021. Early studies have shown that the prominent display of carbon emissions can have positive environmental impact by nudging consumers to choose flights that XXX \cite{amenta_adding_2020}\cite{sanguinetti_nudging_2022}.
    
    %However, different methodologies are used to estimate carbon emissions, leading to significant differences in total estimated carbon emissions. For instance, the development team of \textit{Google Flights} changed the calculation methodology to account for direct-air carbon emissions only, disregarding non-carbon emissions \cite{ali_commit_2022}. This has drawn criticism of \textit{"airbrushing emissions"} by environmental non-governmental organizations \cite{hern_google_2022}\cite{rowlatt_google_2022}.
    
    %Bombardier published an environmental product declaration (LCA) in 2022: \cite{noauthor_challenger_2022}
    
\section{Methods}

    To obtain the highest-accuracy prospective life-cycle assessment of sustainable aviation to date, a scientific software package for modular hybridization of life-cycle inventories is to be developed. It will enable the extension of process-based inventories, using either a tiered hybrid, path-exchange, matrix augmentation or integrated hybrid \cite{hauschild_life_2015}\cite{hauschild_life_2018}\cite{crawford_hybrid_2018} approach. To avoid double-counting of emissions, a heuristic approach based on Agez et al. \cite{agez_lifting_2019}\cite{agez_correcting_2022}\cite{agez_hybridization_2020} will be used initially. The most promising candidates for hybridization are the Ecoinvent (version 3.9) process-based inventory database and the Exiobase (version 3) input-output multi-regional environmentally extended supply-use  table and input-output table. The prototype package \texttt{pylcaio} by Agez et al. \cite{noauthor_pylcaio_2022} will provide a starting point.
    
    In line with existing scientific software for life-cycle assessment, the package will be written in \texttt{Python}. To obtain high computational efficiency on cluster hardware but concurrently enable practitioners to run calculations on consumer hardware, matrix manipulations (matrix multiplication, row-wise and column-wise manipulations, persistence of matrices to disk) will be implemented using appropriate packages, such as \texttt{sparse} \cite{abbasi_sparse_2018}. If applicable, legacy matrix multiplication algorithms will be optimized using novel approaches based on reinforcement learning introduced by Fawzi at al. in 2022 \cite{fawzi_discovering_2022}.
    
    The software package will implement the \texttt{Brightway} API to make use of existing database import functionality and function as part of the envisioned computational engine of the \textit{WISER Digital Ecosystem}. To conform with best practices for documenting scientific software \cite{lee_ten_2018}, logic and schemas will be documented using the \texttt{Sphinx}\footnote{\url{https://www.sphinx-doc.org/}} documentation generator and link to relevant literature.
    
    Using this toolkit, the case study will be conducted using prospective \cite{sacchi_prospective_2022} life-cycle assessment using a hybridized inventory \cite{crawford_hybrid_2018}. Inventory analysis will be guided by global sensitivity analysis based on Monte-Carlo methods developed and implemented by Kim et al. \cite{kim_global_2022}. Information on the future economy and electricity mix will be integrated using methods developed by Sacchi et al. \cite{sacchi_prospective_2022}.

\section{Planned Journal Publications}

    For additional context on the timeline of the publications, including mandatory WISER deliverables and proposed work packages, compare the timeline at \url{https://phd.weinold.ch/timeline}.
    
    \begin{enumerate}
    \setcounter{enumi}{0}
        \item Carbon (Greenhouse Gas) Accounting with Life-Cycle Software \\ Target journals: Journal of Cleaner Production (IF(2020) = 9.297), Journal of Industrial Ecology (IF(2020) = 6.946), International Journal of Life Cycle Assessment (IF(2021) = 5.257)
    \end{enumerate}
    % compare \cite{bourgault_documentation_2021}
    
    \begin{enumerate}
    \setcounter{enumi}{1}
        \item Methodology of Modular Hybridization of Life-Cycle Inventory Data \\ Target journals: Journal of Cleaner Production (IF(2020) = 9.297), Journal of Industrial Ecology (IF(2020) = 6.946), International Journal of Life Cycle Assessment (IF(2021) = 5.257)
    \end{enumerate}
    
    \begin{enumerate}
    \setcounter{enumi}{2}
        \item Prospective Hybrid Life-Cycle Assessment of Sustainable Aviation \\ Target journals: Nature Climate Change (IF(2020) = 21.72), Energy and Environmental Science (IF(2020) = 38.53)
    \end{enumerate}
    
\section{Research in the Context of the WISER Flagship}
    
    The WISER InnoSuisse flagship project\footnote{\url{https://www.innosuisse.ch/inno/de/home/forderung-fur-schweizer-projekte/flagship-initiative/15-flagships.html}} aims to create \textit{"a digital ecosystem that simplifies the access to GHG knowledge"}. Stakeholders, both from industry and the public sector, will be provided an interactive dashboard of their emissions. They will further be provided a simplified way of interacting with life-cycle assessment software using different greenhouse gas assessment frameworks as well as different life-cycle inventory databases, thus eliminating the need for detailed domain knowledge not only of life-cycle assessment, but of the schema of different databases and of the technical infrastructure required to perform extensive calculations. This is facilitated through the deployment of a web service accessible through an application programming interface (API). As the core of sub-project 2 of WISER, the \texttt{Brightway} package will serve as the computational engine of the web service.
    
    Work proposed as part of this doctoral research plan will greatly benefit WISER stakeholders. Hybrid life-cycle assessment will provide more accurate environmental impact assessment, thereby improving confidence in the results. Work defined in the mandatory WISER deliverables of sub-project 2 will in turn greatly benefit the doctoral research. The deployment of \texttt{Brightway} as a web service will provide platform-independent high-performance computing capabilities. Improvements in the technical setup of the documentation will facilitate better translation of scientific code into scientific publications. The implementation of the WISER API will allow for database-agnostic magic.
    
\section{Research in Context of the Department}

    The proposed development of a modular hybridization tool at part of the package \texttt{Brightway} builds on the strong tradition of life-cycle assessment in the Group for Technology Assessment of the Laboratory for Energy Analysis at the Paul Scherrer Institute \footnote{\url{https://www.psi.ch/en/ta/life-cycle-assessment}}. More specifically, life-cycle inventory hybridization is the logical next step connecting work by Aleksandra Kim on global sensitivity analysis of life-cycle assessment \cite{kim_aleksandra-kimgwp_uncertainties_2022}\cite{kim_aleksandra-kimgsa_framework_2021}\cite{paulillo_influential_2021} and work by Romain Sacchi on prospective life-cycle analysis based on integrated assessment models \cite{noauthor_premise_2022}\cite{sacchi_prospective_2022}. The former enables practitioners to understand the drivers of uncertainty, thereby guiding additional life-cycle inventory data collection efforts. However, as illustrated in the introduction, a complete collection of data at the process level is unfeasible. The latter provides the basis for the assessment of future technologies. Together, they provide the basis for prospective hybrid-life cycle assessment. 
    
    The choice of aviation as the main case study of the proposed tools reflects the importance of the topic laid out in the introduction. It builds on and supplements existing work in the Technology Assessment group by Zipeng Liu and Tom Terlouw \cite{terlouw_large-scale_2022}, as well as recent work by Saad \cite{saad_synthetic_2022} and unpublished work by Sacchi et al. \cite{sacchi_climate-neutral_2022}. It further enables the development of the MAVT lectures \textit{Basics of Air Transport (Aviation I)} and \textit{Management of Air Transport (Aviation II)} to include a stronger focus on sustainable aviation.
    
    Due to the multi-method approach combining software development and mathematical optimization with hybrid life-cycle assessment, a number of collaboration opportunities have been identified:
	
	The biggest academic collaboration opportunity identified is the \textit{Cambridge Aviation Impact Accelerator}\footnote{\url{https://www.aiazero.org/}}, founded in 2020 as a collaborative effort of the Whittle Laboratory and Clarence House. Their Resource to Climate Comparison Evaluator (RECCE) \cite{noauthor_recce_2022} provides an overview of the environmental impact of proposed aviation technologies. Similar to the results envisioned for the sustainable aviation case-study proposed as part of this research project, its methodology has not been disclosed. The center takes a holistic approach to sustainable aviation, including a focus on fuels, aircraft and propulsion technologies, airports, non-CO2 emissions, network modelling, broader economic and policy context, and model design. Research initiatives related to sustainable aviation identified in Switzerland and the European Union usually focus on narrow aspects of efficiency improvements (such as the sub-projects of the EU CleanSky2\footnote{\url{https://www.clean-aviation.eu/clean-sky-2}} program) or on sustainable fuels only (such as the call for research into sustainable fuels\footnote{\url{https://www.bfe.admin.ch/bfe/en/home/research-and-cleantech/funding-program-sweet/calls-for-proposals-overview/sweet-call-2-2022.html}} as part of the SWEET (SWiss Energy research for the Energy Transition) program).
    
    In addition, the Sustainability Commission\footnote{\url{https://mavt.ethz.ch/the-department/organization/bodies-and-committees.html}} of the Department of Mechanical Engineering at ETH Zurich has recently published a draft of their white paper on air transport. The commission suggest the founding of  \textit{"a scientific coalition of the international scientific societies and academic institutions for climate-neutral aviation"} \cite{mazzotti_air_2022}. The details of the proposal and any further action on the part of the the university, the department and research are currently pending further decisions by the commission. However, a supporting role, perhaps providing specific insights or providing support during computational tasks, is envisioned.

\section{Progress to Date}

    As per October 2022, the following sub-tasks have been started and/or successfully completed:
    
    \begin{enumerate}
        \item Collection and initial assessment of input/output databases and supply chain information for hybridization. This includes the following databases: \texttt{Exiobase}, \texttt{IEDC}, \texttt{global-fossil-fuel-supply-chain}\footnote{\url{https://github.com/Lkruitwagen/global-fossil-fuel-supply-chain/}}
        \item
            Implementation of \texttt{pylcaio} and \texttt{pymrio} functionality in \texttt{Brightway}.
            \begin{enumerate}
                \item
                    Platform independent testing and debugging setup of \texttt{ecospold2matrix}, \texttt{pymrio}, \texttt{pylcaio} \footnote{\url{https://github.com/michaelweinold/bw_hybrid/blob/master/dev/notebooks}}.
                \item
                    Performance assessment and improvement of existing matrix manipulation and multipilcation algorithms (compare figure \ref{fig:performance}).
                \item
                    Detailed development plan for the refactoring, documentation and implementation of \texttt{pylcaio} functionality in \texttt{Brightway} \footnote{\url{https://github.com/michaelweinold/bw_hybrid/tree/master/.markwhen}}
            \end{enumerate}
        \item
            Migration and upgrade of the existing \texttt{Brightway} API documentation to \texttt{Jupyter Book} \footnote{\url{https://github.com/brightway-lca/brightway-training}} (with \texttt{Thebe} integration) \footnote{\url{https://github.com/brightway-lca/brightway-training/issues/1}} and \texttt{Sphinx} via \url{readthedocs.org}.  \footnote{\url{https://github.com/brightway-lca/brightway-documentation}}. This ensures compliance with the  \textit{Diataxis} documentation framework \footnote{\url{https://diataxis.fr/}}. User-oriented documentation and training materials are now interactive, facilitating. Status: working prototypes completed as part of the \textit{Brightcon 2022} Hackathon \footnote{\url{https://2022.brightcon.link/}}. Further improvements (as part of WISER) pending involvement of external technical writer.
        \item
            Upgrade of \texttt{Brighway} packaging/delivery from \texttt{PyPi} to \texttt{Conda-Forge} to enable platform-specific dependencies \footnote{\url{https://github.com/brightway-lca/brightway2/issues/47}}. This enables user-friendly setup on macOS running on M1 processors with AArch64 ("ARM64") architecture \footnote{\url{https://github.com/brightway-lca/brightway2/issues/46}}. Previously, platform-specific sparse linear algebra solvers required high levels of user intervention during setup with high failure rates \footnote{\url{https://github.com/LCA-ActivityBrowser/activity-browser/issues/705}}. Status: New setup tested, respective GitHub issues closed. Merge of \texttt{conda} recipes into the \texttt{conda-forge} repository completed for first core packages, other packages pending.
    \end{enumerate}

\begin{figure}[h!]
	\centering
	\includegraphics[width=\textwidth]{figures/performance.png}
	\caption{Average runtime of different sparse matrix row-wise manipulation algorithms operating on two different matrices. The most efficient algorithm (\textit{"data mask"}) achieves a runtime improvement of three orders of magnitude over the current implementation in \texttt{pylcaio}. Similar improvements are expected for other parts of the hybridization, based on the current understanding of the code. An overall improvement of at least one, possibly two orders of magnitude Measured using \texttt{\%timeit} (number of loops determined automatically, dependent on convergence of measurement results) under Python 3.10. on an Apple M1 Max platform with 32GB RAM \cite{weinold_github_2022}.}
	\label{fig:performance}
\end{figure}

\newpage
\section{Work Packages}

    Work packages WP1.X are designed to address the aforementioned issues with current solutions for the hybridization of life-cycle inventory. They will also provide the toolkit for investigation of specific technologies in work packages WP3.X. In themselves, they will provide a valuable contribution to the scientific community practicing life-cycle assessment. Building on existing work of the \texttt{Brightway} package, they will be reproducible, XXX, XXX.
    
    Work packages WP2.X are designed to meet the mandatory WISER deliverables of sub-project 2. The improvements in the computational infrastructure are also targeted at streamlining computationally heavy tasks in work packages WP3.X. They will further provide a simplified framework for the technical documentation of scientific code, which will speed up any publications resulting from work packages WP1.X and WP3.X \footnote{Compare, for instance, the \texttt{pvlib} documentation \url{https://pvlib-python.readthedocs.io/en/stable/reference/generated/pvlib.atmosphere.gueymard94_pw.html}}.
    
    Work packages WP3.X build on the toolkit provided by work packages WP1.X and WP2.X. They are designed to answer specific questions related to the field of sustainable aviation. Out of all work packages, they provide the best opportunity for supervision of master's students.

    Note that generic WISER deliverables, such as annual reports and the mid-term review, are not associated with any one work package. For a full list of WISER deliverables, compare "Subproject 2: Computation \& Assessment Toolbox" in the appendix.

    \begin{table}[H]
        \centering
        
        \begin{tabularx}{\linewidth}{
            |>{\hsize=0.25\hsize}X
            |>{\hsize=1.\hsize}X
            |>{\hsize=1.\hsize}X
            |>{\hsize=1.\hsize}X
            |>{\hsize=0.75\hsize}X|
          } % https://tex.stackexchange.com/a/249043
            \hline
                \textbf{\#} & \textbf{Work Package} & \textbf{WISER} & \textbf{Risks} & \textbf{Duration}
            \\
            \hline
        \end{tabularx}
        
        \begin{tabularx}{\linewidth}{
            |>{\hsize=0.25\hsize}X
            |>{\hsize=1.\hsize}X
            |>{\hsize=1.\hsize}X
            |>{\hsize=1.\hsize}X
            |>{\hsize=0.75\hsize}X|
          } % https://tex.stackexchange.com/a/249043
            \hline
                WP1.1
            &
                LCI Hybridization 
            &
                D2.6 (del. month 9) \newline D2.9 (del. month 42)
            &
                affects WP1.2
            &
                11 - 12 months
            \\
            \hline
        \end{tabularx}
        
    \end{table}
    \vspace*{-9pt}
    
    Life-cycle inventory hybridization (augmentation of foreground processes from process-based life-cycle assessment database with additional information from input-output-based database) to increase accuracy of life-cycle assessment. In scope: \textit{Ecoinvent} and \textit{Exiobase}. For other databases, compare W1.3. Associated risks: Heuristics remain the only viable option to avoid double-counting of emissions. A high degree of manual intervention remains a requirement in hybridization, impeding the simple adoption of other databases in WP1.2. No critical risk to downstream work packages.

    \begin{table}[H]
        \centering
        \begin{tabularx}{\linewidth}{
            |>{\hsize=0.25\hsize}X
            |>{\hsize=1.\hsize}X
            |>{\hsize=1.\hsize}X
            |>{\hsize=1.\hsize}X
            |>{\hsize=0.75\hsize}X|
          } % https://tex.stackexchange.com/a/249043
            \hline
                WP1.2
            &
                \texttt{bw\_hybrid} Integration
            &
                D2.6 (del. month 9) \newline D2.9 (del. month 42)
            &
                dependent on WP1.1
            &
                $\sim$ 6 months
            \\
            \hline
        \end{tabularx}
    \end{table}
    \vspace*{-9pt}
    
    Systematic setup and completion of integration and unit tests for code originating from WP1.1. Also includes relevant refactoring required for integration with \textit{Brightway3} and standards proposed in the \textit{Brightway Strategic Development Plan}.  In scope: 80-100\% test coverage, performance assessments of matrix operations and documentation. Associated risks: None, the tests are a requirement for reproducible results. Most tests can be simple \texttt{assert} statements on before-and-after sample matrices.

    \begin{table}[H]
        \centering
        \begin{tabularx}{\linewidth}{
            |>{\hsize=0.25\hsize}X
            |>{\hsize=1.\hsize}X
            |>{\hsize=1.\hsize}X
            |>{\hsize=1.\hsize}X
            |>{\hsize=0.75\hsize}X|
          } % https://tex.stackexchange.com/a/249043
            \hline
                WP1.3
            &
                LCI Hybridization Database Evaluation
            &
                N/A
            &
                affects no other WP
            &
                $\sim$ 6 months
            \\
            \hline
        \end{tabularx}
    \end{table}
    \vspace*{-9pt}
    
    Continuation of initial assessment of input-output databases for integration with existing hybridization infrastructure developed in W1.1/W1.2. If possible, code from WP1.1 will be amended to offer a greater range of options for hybridization. Higher-resolution supply chain databases will be considered. In scope: \textit{Industrial Ecology Dashboard} and associated databases, various regional input-output databases. Associated risks: If nomenclature and conventions of databases differ too greatly, a simple addition to existing \texttt{bw\_hybrid} functionality will be beyond the scope of the work package. In this case, required steps can be documented to ensure that other interested parties can contribute. Even if no other databases can be added, no other downstream work packages are affected.
    
    \begin{table}[H]
        \centering
        \begin{tabularx}{\linewidth}{
            |>{\hsize=0.25\hsize}X
            |>{\hsize=1.\hsize}X
            |>{\hsize=1.\hsize}X
            |>{\hsize=1.\hsize}X
            |>{\hsize=0.75\hsize}X|
          } % https://tex.stackexchange.com/a/249043
            \hline
                WP2.1
            &
                \texttt{Brightway} Documentation Upgrade
            &
                D2.6
            &
                affects no other WP
            &
                3 months (core)
            \\
            \hline
        \end{tabularx}
    \end{table}
    \vspace*{-9pt}
    
    Upgrade of the existing Brightway documentation in line with the \textit{Brightway Strategic Development Plan}. Release of the documentation for comment to WISER stakeholders and other sub-project leaders. In scope: Upgrade of the documentation to reflect changes as part of WISER sub-project 2, in accordance with \texttt{BEP003}\footnote{\url{https://github.com/brightway-lca/enhancement-proposals/blob/main/proposals/0003.md}}. Associated risks: None identified.
    
    \begin{table}[H]
        \centering
        \begin{tabularx}{\linewidth}{
            |>{\hsize=0.25\hsize}X
            |>{\hsize=1.\hsize}X
            |>{\hsize=1.\hsize}X
            |>{\hsize=1.\hsize}X
            |>{\hsize=0.75\hsize}X|
          } % https://tex.stackexchange.com/a/249043
            \hline
                WP2.2
            &
                GHG Standards in \texttt{Brightway}
            &
                D2.5, M2.7
            &
                affects no other WP
            &
                $\sim$ 6 months
            \\
            \hline
        \end{tabularx}
    \end{table}
    \vspace*{-9pt}
    
    Analysis and comparison of different greenhouse gas assessment standards (carbon footprinting standards) in the Brightway life-cycle assessment software framework. This includes existing implementations and potential differences to novel implementations. Implementation of multiple nomenclature conventions. In scope (as per deliverable): ISO 14067, PAS2050, GHG protocol (scope 1, 2 and 3). Associated risks: None identified.
    
    \begin{table}[H]
        \centering
        \begin{tabularx}{\linewidth}{
            |>{\hsize=0.25\hsize}X
            |>{\hsize=1.\hsize}X
            |>{\hsize=1.\hsize}X
            |>{\hsize=1.\hsize}X
            |>{\hsize=0.75\hsize}X|
          } % https://tex.stackexchange.com/a/249043
            \hline
                WP2.3
            &
                WISER SP1/SP2 API
            &
                D2.7, M2.5
            &
                mandatory deliverable \newline affects all other SPs \newline affects WP2.4
            &
                $\sim$ 6 months
            \\
            \hline
        \end{tabularx}
    \end{table}
    \vspace*{-9pt}
    
    Development of an API interface for \texttt{Brightway} to be used in conjunction with the proposed web-service. Provides an interface to the WISER API developed by sub-project 1. In scope: adaption of existing \texttt{Brightway} infrastructure to allow for full interaction with (currently in development) sub-project 1 API calls. Most likely includes basic LCA calculations, including the choice of dataset, etc. Associated risks: The required changes might be difficult to implement or take an inordinate amount of time. This is also an area of software development in which I have the least experience. Summarily, this work package was identified to carry the highest associated risks, considering also the small benefit rendered to downstream work packages WP3.X, which are more relevant to pulblications vital to the progress of my doctoral studies.
    
    \begin{table}[H]
        \centering
        \begin{tabularx}{\linewidth}{
            |>{\hsize=0.25\hsize}X
            |>{\hsize=1.\hsize}X
            |>{\hsize=1.\hsize}X
            |>{\hsize=1.\hsize}X
            |>{\hsize=0.75\hsize}X|
          } % https://tex.stackexchange.com/a/249043
            \hline
                WP2.4
            &
                \texttt{Brightway} Webservice
            &
                D2.8, M2.6
            &
                mandatory deliverable \newline affects all other SPs 
            &
                $\sim$ 4 months
            \\
            \hline
        \end{tabularx}
    \end{table}
    \vspace*{-9pt}
    
    Enhancement of existing \texttt{Brightway} functionality to provide a web-service to be used as the core "computational engine" of the WISER project. In scope: Streamlining the setup of the packages on a virtual private server and the associated documentation (compare WP2.1). Potentially a \texttt{Docker} container of the required packages. Associated risks: Negligible. Existing infrastructure \footnote{Compare the current Brightway Jupyter-Hub server\url{https://hub.brightway.dev}} and documentation provide an excellent basis. The associated benefit for downstream work packages WP3.X is large, considering the high degree of manual intervention that was still required in the work of Aleksandra Kim \cite{paulillo_influential_2021}.

    \begin{table}[H]
        \centering
        \begin{tabularx}{\linewidth}{
            |>{\hsize=0.25\hsize}X
            |>{\hsize=1.\hsize}X
            |>{\hsize=1.\hsize}X
            |>{\hsize=1.\hsize}X
            |>{\hsize=0.75\hsize}X|
          } % https://tex.stackexchange.com/a/249043
            \hline
                WP3.1
            &
                Sustainable Aviation: \newline Data Collection
            &
                N/A
            &
                affects WP3.2
            &
                $\sim$ 6 months
            \\
            \hline
        \end{tabularx}
    \end{table}
    \vspace*{-9pt}
    
    This includes initial sensitivity/contribution analysis of the underlying data to guide targeted data collection. Early results are to be benchmarked against new industry publications, such as \cite{noauthor_challenger_2022}. In scope: TBD. Associated risks: TBD.
    
    \begin{table}[H]
        \centering
        \begin{tabularx}{\linewidth}{
            |>{\hsize=0.25\hsize}X
            |>{\hsize=1.\hsize}X
            |>{\hsize=1.\hsize}X
            |>{\hsize=1.\hsize}X
            |>{\hsize=0.75\hsize}X|
          } % https://tex.stackexchange.com/a/249043
            \hline
                WP3.2
            &
                Sustainable Aviation: \newline Hybrid LCA
            &
                N/A
            &
                N/A
            &
                $\sim$ 12 months
            \\
            \hline
        \end{tabularx}
    \end{table}
    
    This includes the (potentially prospective, based on \texttt{premise}) life-cycle assessment based on a hybridized inventory. Scope: TBD. Risks: TBD.
    
\section{Time Schedule}

    The proposed schedule associated with this research plan can be viewed as an interactive \textit{Markwhen}\footnote{\url{https://github.com/kochrt/markwhen}} timeline at \url{https://phd.weinold.ch/timeline}.

\newpage
\section{Appendix}

 % https://docs.google.com/spreadsheets/d/14gmD96CXOYMKjaUK2KIvosC4LcfpzAc-_INdCU4LAro/edit#gid=1019504979

    \subsection{Details of Systematic Literature Review Methodology}
    
        The \textit{SALSA} (\textbf{S}earch, \textbf{A}ppraisa\textbf{L}, \textbf{S}ynthesis, \textbf{A}nalysis) framework rationalization introduced by Grant et al. \cite{grant_typology_2009} was utlized in the context of the systematic literature review. Search was conducted through the \textit{Elsevier Scopus Advanced Search} engine \footnote{\url{https://www.scopus.com/search/form.uri?display=advanced}}. Out of the small body of search engines supporting Boolean queries, it was chosen over \textit{Dimensions} \footnote{\url{https://app.dimensions.ai/discover/publication}} and \textit{The Lens} \footnote{\url{https://www.lens.org/lens/search/scholar/structured}} for its ease of use and larger body of indexed publications. Inclusion and exclusion criteria listed in section \ref{subsub_inclusion_exclusion} were used to contruct Boolean search queries listed in section \ref{sub_boolean} to significantly narrow the scope of publications for appraisal. 
        
        \subsubsection{Inclusion/Exclusion Criteria}
        \label{subsub_inclusion_exclusion}
        
            To reduce the number of false positive results, a list of relevant journal subject areas was compiled:
            
\begin{code_search}
F1: PHYS OR ENER OR CENG OR CHEM OR COMP OR EART OR ENVI
F2: F1 OR ECON OR BUSI OR SOCI
F3: PHYS OR ENER OR CENG OR CHEM OR COMP OR EART OR ENVI
\end{code_search}

        
            \begin{table}[H]
                \centering
                \begin{tabularx}{\textwidth}{| X | X |}
                    \hline
                \multicolumn{2}{|l|}{\textbf{(F1.1) Hybrid LCA methods}}  \\
                    \hline
                    \textit{inclusion criteria} & \textit{exclusion criteria} \\
                    \hline
                        hybridization methodology is primary focus \newline
                        process-based life-cycle assessment (PLCA) \newline
                        multi-regional input-output tables (MRIO) \newline
                        environmentally extended IO analysis (EEIOA) \newline
                        specific databases: EXIOBASE, ecoinvent \newline
                        focus on truncation errors and double-counting
                    &
                        journal subject areas
                    \\
                    \hline
                \multicolumn{2}{|l|}{\textbf{(F2.2) Prospective hybrid LCA methods}}  \\
                    \hline
                    \textit{inclusion criteria} & \textit{exclusion criteria} \\
                    \hline
                        all inclusion criteria applied in (2.1) \newline
                        prospective/future hybrid LCA
                    &
                        all exclusion criteria applied in (2.1)
                    \\
                    \hline
                \multicolumn{2}{|l|}{\textbf{(F2.1) LCA specific to (sustainable) aviation}}  \\
                    \hline
                    \textit{inclusion criteria} & \textit{exclusion criteria} \\
                    \hline
                        LCA used as primary method  \newline
                    &
                        LCA target domain too specific \newline (e.g. "aviation catering") \newline
                        fuel input too specific \newline (e.g. "bioethanol production from expired cookies") \newline
                        journal subject areas
                    \\
                    \hline
                \multicolumn{2}{|l|}{\textbf{(F2.2) hybrid LCA specific to (sustainable) aviation}}  \\
                    \hline
                    \textit{inclusion criteria} & \textit{exclusion criteria} \\
                    \hline
                        hybrid LCA used as primary method  \newline
                    &
                        all exclusion criteria applied in (1.1) \newline
                        primary focus on hybrid-electric propulsion \newline
                        (e.g. "Feasibility Study of Hybrid Propulsion")
                    \\
                    \hline
                \multicolumn{2}{|l|}{\textbf{(F3.1) Impact of different GHG standards/implementations in LCA}}  \\
                    \hline
                    \textit{inclusion criteria} & \textit{exclusion criteria} \\
                    \hline
                        comparison of specific assessment frameworks in the context of LCA\newline
                        (i.e. ISO 14040-14044, GHG Protocol, ILCD Handbook, ISO 14067, CDP, GRI, SBTi, PEF/OEF, ESG)
                    &
                        narrow focus on any one framework from the inclusion criteria \newline
                        LCA not in research context
                    \\
                    \hline
                \end{tabularx}
                \caption{Inclusion and exclusion criteria for systematic literature review. Where applicable and possible, these were translated to a Boolean search logic. Compare also section \ref{sub_boolean}. The list of relevant subject areas was compiled by collecting metadata of false positive search results.}
                \label{tab_inclusion_exclusion}
            \end{table}
    
\newpage
\bibliographystyle{IEEEtran}
\bibliography{references}

\end{document}
