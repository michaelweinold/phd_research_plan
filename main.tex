\documentclass{article}
% formatting
\usepackage{aux/style_files/arxiv}
\usepackage[utf8]{inputenc} % allow utf-8 input
\usepackage[T1]{fontenc}    % use 8-bit T1 fonts
% cross-referencing
\usepackage{hyperref}
% tables
\usepackage{booktabs}
\usepackage{tabularx}
\usepackage{caption} 
\captionsetup[table]{skip=10pt}
% fonts
\usepackage{amsfonts}
\usepackage{microtype}
% color
\usepackage{color}
\definecolor{verbgray}{gray}{0.9}
% figures
\usepackage{graphicx}
% enumeration
\usepackage{enumitem}
% embedd pdf files
\usepackage{pdfpages}

% code
\usepackage{listings} % https://tex.stackexchange.com/a/53951
\lstnewenvironment{code_search}{
    \lstset{
        backgroundcolor=\color{verbgray},
        frame=single,
        basicstyle=\ttfamily,
        %columns=fullflexible,
        xleftmargin=0cm,
    }
}{}
\setcounter{secnumdepth}{5}

\usepackage{footnote}
\makesavenoteenv{tabular}

\title{Life-Cycle Assessment for Sustainable Aviation}


\author{
    \href{https://orcid.org/0000-0003-4859-2650}
    {\includegraphics[scale=0.06]{aux/figures/orcid.pdf}
    \hspace{1mm}
    Michael Weinold} \\
	Paul Scherrer Institut\\
	Laboratory for Energy Systems Analysis\\
	Group for Technology Assessment\\
	OSHA/D22\\
    5232 Villigen PSI \\
    Switzerland \\
	\texttt{michael.weinold@psi.ch} \\
}

\date{28. July 2022}

\renewcommand{\headeright}{Doctoral Research Plan}
\renewcommand{\undertitle}{Doctoral Research Plan}

%%% Add PDF metadata to help others organize their library
%%% Once the PDF is generated, you can check the metadata with
%%% $ pdfinfo template.pdf
\hypersetup{
pdftitle={Doctoral Research Plan},
pdfauthor={Michael Weinold},
}

\begin{document}

\includepdf[pages=-]{attachments/research_plan_flat.pdf}

\maketitle

\begin{abstract}
	\cite{becattini_role_2021}
\end{abstract}

Attachments:
WISER timeline, WISER sub-project 2 description

\section{Introduction}

\cite{prussi_corsia_2021}

how are sustainable aviation fuels defined -> CORSIA \cite{prussi_corsia_2021}

    Carbon emissions estimations for individual flights have been available from major flight search providers, including \textit{Google Flights} \cite{holden_google_2021}, \textit{Kayak} \cite{noauthor_kayak_2021} and \textit{SkyScanner} \cite{crosthwaite_how_2021} from 2021. Early studies have shown that the prominent display of carbon emissions can have positive enironmental impact by nudging consumers to choose flights that XXX \cite{amenta_adding_2020}\cite{sanguinetti_nudging_2022}.
    
    However, different methodologies are used to estimate carbon emissions, leading to significant differences in total estimated carbon emissions. For instance, the development team of \textit{Google Flights} changed the calculation methodology to account for direct-air carbon emissions only, disregarding non-carbon emissions \cite{ali_commit_2022}. This has drawn criticism of \textit{"airbrushing emissions"} by environmental non-governmental organizations \cite{hern_google_2022}\cite{rowlatt_google_2022}.

\section{Hybrid Life Cycle Assessment}

	What is life-cycle assessment?
	Why is its use in aviation a novelty?
	
	Some recent publications have completely disregarded the impact of life-cycle emissions under the assumption that these would be negligible \cite{brazzola_definitions_2022}.
	
	Why is the use of hybrid life-cycle inventory methods in aviation a novelty?
	
\section{USP}
	
    Existing knowledge within PSI, for instance the recent thesis by Saad \cite{saad_synthetic_2022} and unpublished work by \cite{sacchi_climate-neutral_2022}.

\section{Ideas}

    Connect this to "M2.1: Publication analysing impact of multiple standards and standards implementations in LCA"

    Connect this to "RECCE 2035: Resource to Climate Comparison Evaluator"

	Chris: research and development phase often missed completely in LCA
	Chris: Airbus is using Brightway in future aircraft design
	Chris: knowledge in the group on sustainable fuels (Tom Terlouw?)
	Laura: Cambridge Aviation Impact Accelerator meeting in September - maybe meet before that?

\section{Questions}

How do the results from Romain's preprint \cite{sacchi_climate-neutral_2022} compare against the new ICAO paper \cite{prussi_corsia_2021}

\newpage
\section{Systematic Literature Review}


    \subsection{Hybrid LCA}
        
        Hybrid life-cycle assessment refers to the combination of two different approaches to building inventory data. 
        
        First ideas in the 80s. The first significant mention of hybrid LCA in review by Lenzen in 2000 \cite{lenzen_errors_2000}.
            
        Why are differences in PLCA so large? One reason: different choice of system boundaries ( compare 2. Method and data 2.1. Hybrid life cycle assessment in \cite{teh_hybrid_2017}
        
        The difference in results between a purely process-based and a hybrid life-cycle assessment is naturally dependent on the parameters listed in section. FIGURE HERE!
        
        The only review of hybrid life-cycle assessment methods identified is provided by Crawford et al. in 2018 \cite{crawford_hybrid_2018}. \cite{}
        
        \begin{table}[htbp]
            \centering
            \begin{tabularx}{\textwidth}{| X | X | X | X |}
                \hline
                    \textbf{Study} & \textbf{Sector} & \textbf{Methodology} & \textbf{Indicators} \\
                \hline
                    Li et al., 2019 \cite{li_economic_2019} & Resource Extraction & IO + PLCI & CO2, SO2, NOx, soot \\
                \hline
                    Li et al., 2020 \cite{li_life_2020} & Agriculture & EIO + PLCI, MC-based sensitivity analysis & 10 from TRACI v2.1 \\ % . The results show that environmental impact from EIO-based inventory contributes a meaningful fraction of the impact for ozone depletion (67%), respiratory effects (42%), fossil fuel depletion (38%), and smog (28%) (as opposed to process-based inventory)
                \hline
                    Krishnan et al., 2004 \cite{krishnan_using_2004} & Semiconductors & bespoke IO + PLCA & electricity \\
                \hline
                    Wang et al., 2014 \cite{wang_hybrid_2014} & Semiconductors & EIO + LCI (stochiometric approach for chemicals) & human toxicity (cancer), water eutrophication, GWP, etc. \\
                \hline
                    Stokes et al., 2008 \cite{stokes_energy_2009} & Water Supply & EIOLCA tool \footnote{\url{http://www.eiolca.net/}} & CO2, SO2, NOx, soot, etc. \\
                \hline
                    Murray et al., 2008 \cite{murray_hybrid_2008} & Wastewater Treatment & EIOLCA tool \footnote{\url{http://www.eiolca.net/}} & CO2, SO2, NOx, soot, etc. \\
                \hline
                    Teh et al., 2017 \cite{teh_hybrid_2017} & Raw Materials Production & IELab tool \footnote{\url{https://ielab.info/}} & CO2 \\
                \hline
                    Facanha et al., 2006 \cite{horvath_environmental_2006} & Transportation (Land) & EIO + PLCI & CO2, NOx, PM10, CO \\
                \hline
                    Ewing et al., 2011 \cite{ewing_insights_2011} & Transportation (Sea) & EIOLCA tool \footnote{\url{http://www.eiolca.net/}} & CO2eq \\
                \hline
                    Martinez-Corona et al., 2016 \cite{martinez-corona_hybrid_2017} & Electricity (Geotherm.) & THEMIS tool \cite{gibon_methodology_2015} & GWP, etc. \\
                \hline
                    Vasan et al., 2014 \cite{vasan_carbon_2014} & Electronics Manufacturing & EIO + PLCI & CO2eq \\
                \hline
                    Arvesen et al., 2014 \cite{arvesen_life_2014} & Electricity (Wind) & EIO + PLCI & multiple \\
                \hline
                    Yao et al., 2014, \cite{yao_hybrid_2014} & PV Manuf. & EIO + PLCI & multiple \\
                \hline
                    Zhao et al., 2019 \cite{zhao_comparative_2019} & Li-Ion Battery Manuf. & detailed (follow-up) & multiple \\
                \hline
                    Wu et al., 2019 \cite{wu_assessing_2019} & Transportation (Land) & THEMIS tool \cite{gibon_methodology_2015} & CO2eq \\
                \hline
                    Ramaswami et al., 2008 \cite{ramaswami_demand-centered_2008} & Urban Emissions & generic & CO2eq \\
                \hline
                    Chang et al., 2014 \cite{chang_shale--well_2014} & Resource Extraction & generic? & CO2eq, SO2, NOx, CH4 \\
                \hline
            \end{tabularx}
            \caption{Selection of environmental impact studies using a hybrid life-cycle assessment methodology. The search query used to list all relevant studies is listed in section \ref{sub_boolean} in the Appendix. Studies excluded are those older than 5 years with less than 5 citations and those describing sectors deemed too specific (e.g. \textit{"Comparative life cycle assessment of drinking straws in Brazil" \cite{zanghelini_comparative_2020}}. Abbreviations: IO = Input Output Table/Database, PLCI = Process-Based Life-Cycle Inventory Database, MC = Monte-Carlo Technique}
        \end{table}

    \subsection{Life-Cycle Assessment of (Sustainable) Aviation}

\newpage
\section{Progress to Date}

    As per October 2022, the following sub-tasks have been started and/or successfully completed:
    
    \begin{enumerate}
        \item collection of input/output databases for hybridization
        \item
            Implementation of \texttt{pylcaio} and \texttt{pymrio} functionality in \texttt{Brightway}.
            \begin{enumerate}
                \item Platform independent testing and debugging setup of \texttt{ecospold2matrix}, \texttt{pymrio}, \texttt{pylcaio}.
            \end{enumerate}
        \item
            Migration and upgrade of the existing \texttt{Brightway} API documentation to \texttt{Jupyter Book} \footnote{\url{https://github.com/brightway-lca/brightway-training}} (with \texttt{Thebe} integration) \footnote{\url{https://github.com/brightway-lca/brightway-training/issues/1}} and \texttt{Sphinx} via \url{readthedocs.org}.  \footnote{\url{https://github.com/brightway-lca/brightway-documentation}}. This ensures compliance with the  \textit{Diataxis} documentation framework \footnote{\url{https://diataxis.fr/}}. User-oriented documentation and training materials are now interactive, facilitating . Status: working prototypes completed as part of the \textit{Brightcon 2022} Hackathon \footnote{\url{https://2022.brightcon.link/}}.
        \item
            Upgrade of \texttt{Brighway} packaging/delivery from \texttt{PyPi} to \texttt{Conda-Forge} to enable platform-specific dependencies \footnote{\url{https://github.com/brightway-lca/brightway2/issues/47}}. This enables user-friendly setup on macOS running on M1 processors with AArch64 ("ARM64") architecture \footnote{\url{https://github.com/brightway-lca/brightway2/issues/46}}. Previously, platform-specific sparse linear algebra solvers required high levels of user intervention during setup with high failure rates \footnote{\url{https://github.com/LCA-ActivityBrowser/activity-browser/issues/705}}. Status: New setup tested, respective GitHub issues closed. Merge of \texttt{conda} recipes into the \texttt{conda-forge} repository completed for first core packages, other packages pending.
    \end{enumerate}

    1. collected databases for hybridization
    2. started refactoring and rewriting the pylcaio code
        2.1. improved memory and computational performance
        2.2. improved documentation
    3. started upgrading the Brightway documentation
    4. started adapting the Brightway computational engine to ARM64 (M1) processors
    5. literature review and collection of LCA studies in aviation
    6. 

\section{Work Packages}



\newpage
\section{Appendix}

 % https://docs.google.com/spreadsheets/d/14gmD96CXOYMKjaUK2KIvosC4LcfpzAc-_INdCU4LAro/edit#gid=1019504979

    \subsection{Details of Systematic Literature Review Methodology}
    
        The \textit{SALSA} (\textbf{S}earch, \textbf{A}ppraisa\textbf{L}, \textbf{S}ynthesis, \textbf{A}nalysis) framework rationalization introduced by Grant et al. \cite{grant_typology_2009} was utlized in the context of the systematic literature review. Search was conducted through the \textit{Elsevier Scopus Advanced Search} engine \footnote{\url{https://www.scopus.com/search/form.uri?display=advanced}}. Out of the small body of search engines supporting Boolean queries, it was chosen over \textit{Dimensions} \footnote{\url{https://app.dimensions.ai/discover/publication}} and \textit{The Lens} \footnote{\url{https://www.lens.org/lens/search/scholar/structured}} for its ease of use and larger body of indexed publications. Inclusion and exclusion criteria listed in section \ref{subsub_inclusion_exclusion} were used to contruct Boolean search queries listed in section \ref{sub_boolean} to significantly narrow the scope of publications for appraisal. 
        
        \subsubsection{Inclusion/Exclusion Criteria}
        \label{subsub_inclusion_exclusion}
        
            To reduce the number of false positive results, a list of relevant journal subject areas was compiled:
            
\begin{code_search}
F1: PHYS OR ENER OR CENG OR CHEM OR COMP OR EART OR ENVI
F2: F1 OR ECON OR BUSI OR SOCI
F3: PHYS OR ENER OR CENG OR CHEM OR COMP OR EART OR ENVI
\end{code_search}

        
            \begin{table}[htbp]
                \centering
                \begin{tabularx}{\textwidth}{| X | X |}
                    \hline
                \multicolumn{2}{|l|}{\textbf{(F1.1) Hybrid LCA methods}}  \\
                    \hline
                    \textit{inclusion criteria} & \textit{exclusion criteria} \\
                    \hline
                        hybridization methodology is primary focus \newline
                        process-based life-cycle assessment (PLCA) \newline
                        multi-regional input-output tables (MRIO) \newline
                        environmentally extended IO analysis (EEIOA) \newline
                        specific databases: EXIOBASE, ecoinvent \newline
                        focus on truncation errors and double-counting
                    &
                        journal subject areas
                    \\
                    \hline
                \multicolumn{2}{|l|}{\textbf{(F2.2) Prospective hybrid LCA methods}}  \\
                    \hline
                    \textit{inclusion criteria} & \textit{exclusion criteria} \\
                    \hline
                        all inclusion criteria applied in (2.1) \newline
                        prospective/future hybrid LCA
                    &
                        all exclusion criteria applied in (2.1)
                    \\
                    \hline
                \multicolumn{2}{|l|}{\textbf{(F2.1) LCA specific to (sustainable) aviation}}  \\
                    \hline
                    \textit{inclusion criteria} & \textit{exclusion criteria} \\
                    \hline
                        LCA used as primary method  \newline
                    &
                        LCA target domain too specific \newline (e.g. "aviation catering") \newline
                        fuel input too specific \newline (e.g. "bioethanol production from expired cookies") \newline
                        journal subject areas
                    \\
                    \hline
                \multicolumn{2}{|l|}{\textbf{(F2.2) hybrid LCA specific to (sustainable) aviation}}  \\
                    \hline
                    \textit{inclusion criteria} & \textit{exclusion criteria} \\
                    \hline
                        hybrid LCA used as primary method  \newline
                    &
                        all exclusion criteria applied in (1.1) \newline
                        primary focus on hybrid-electric propulsion \newline
                        (e.g. "Feasibility Study of Hybrid Propulsion")
                    \\
                    \hline
                \multicolumn{2}{|l|}{\textbf{(F3.1) Impact of different GHG standards/implementations in LCA}}  \\
                    \hline
                    \textit{inclusion criteria} & \textit{exclusion criteria} \\
                    \hline
                        comparison of specific assessment frameworks in the context of LCA\newline
                        (i.e. ISO 14040-14044, GHG Protocol, ILCD Handbook, ISO 14067, CDP, GRI, SBTi, PEF/OEF, ESG)
                    &
                        narrow focus on any one framework from the inclusion criteria \newline
                        LCA not in research context
                    \\
                    \hline
                \end{tabularx}
                \caption{Inclusion and exclusion criteria for systematic literature review. Where applicable and possible, these were translated to a Boolean search logic. Compare also section \ref{sub_boolean}. The list of relevant subject areas was compiled by collecting metadata of false positive search results.}
                \label{tab_inclusion_exclusion}
            \end{table}
    
    \subsubsection{Boolean Search Logic}
    \label{sub_boolean}
        
        Note that all upper case strings indicate search operators, all lower case strings indicate search terms. Compare the \textit{Scopus Advanced Search} documentation for a list of operator precedence and syntax definitions \footnote{\url{https://service.elsevier.com/app/answers/detail/a_id/11365/}}. For the inclusion and exclusion considerations that informed the logic, compare table \ref{tab_inclusion_exclusion}.
        
        \paragraph{(F1.1) Hybrid LCA methods}
        \ % required for listings to work with \paragraph{}
        
\begin{code_search}
TITLE-ABS-KEY(
    {hybrid life-cycle}
    OR
    {hybrid life cycle}
    OR
    {hybrid LCA}
)
AND TITLE-ABS-KEY(
    *mrio* or *plca* or eeio or hlca
)
AND TITLE-ABS-KEY(
    {truncation} or {double counting} or {double-counting}
)
AND SUBJAREA(
    PHYS OR ENER OR CENG OR CHEM OR COMP OR EART OR ENVI OR
    ECON OR BUSI OR SOCI
)
\end{code_search}

        \paragraph{(F1.2) Prospective (hybrid) LCA methods}
        \ % required for listings to work with \paragraph{}
        
\begin{code_search}
TITLE-ABS-KEY(
    {hybrid life-cycle}
    OR
    {hybrid life cycle}
    OR
    {hybrid LCA}
)
AND TITLE-ABS-KEY(
    prospect*
)
AND SUBJAREA(
    PHYS OR ENER OR CENG OR CHEM OR COMP OR EART OR ENVI
)
\end{code_search}

        \paragraph{(F2.1) LCA specific to (sustainable) aviation}
        \ % required for listings to work with \paragraph{}
            
\begin{code_search}
TITLE-ABS-KEY(
    {life-cycle assessment}
    OR
    {life cycle assessment}
    OR
    {lca}
)
AND TITLE-ABS-KEY(
    sustainable W/3 aviation
)
AND SUBJAREA(
    PHYS OR ENER OR CENG OR CHEM OR COMP OR EART OR ENVI OR
    ECON OR BUSI OR SOCI
)
\end{code_search}

        \paragraph{(F2.2) Hybrid LCA specific to (sustainable) aviation}
        \ % required for listings to work with \paragraph{}
            
\begin{code_search}
TITLE-ABS-KEY(
    {hybrid life-cycle assessment}
    OR
    {hybrid life cycle assessment}
    OR plca OR *mrio* or eeio
)
AND TITLE-ABS-KEY(
    aviation OR {air transport}
)
AND SUBJAREA(
    PHYS OR ENER OR CENG OR CHEM OR COMP OR EART OR ENVI OR
    ECON OR BUSI OR SOCI
)
\end{code_search}

        \paragraph{(F3.1) Impact of different GHG standards/implementations in LCA}
        \ % required for listings to work with \paragraph{}

\begin{code_search}
TITLE-ABS-KEY(
    {hybrid life-cycle assessment}
    OR
    {hybrid life cycle assessment}
    OR plca OR *mrio* or eeio
)
AND TITLE-ABS-KEY(
    aviation OR {air transport}
)
AND SUBJAREA(
    PHYS OR ENER OR CENG OR CHEM OR COMP OR EART OR ENVI OR
    ECON OR BUSI OR SOCI
)
\end{code_search}

\newpage

\bibliographystyle{IEEEtran}
\bibliography{references}

\includepdf[pages=-]{attachments/phd_job_description.pdf}

\end{document}
