\documentclass{article}
% formatting
\usepackage{aux/style_files/arxiv}
\usepackage[utf8]{inputenc} % allow utf-8 input
\usepackage[T1]{fontenc}    % use 8-bit T1 fonts
% cross-referencing
\usepackage{hyperref}
\usepackage{url} 
% tables
\usepackage{booktabs}
% fonts
\usepackage{amsfonts}
\usepackage{microtype}
% figures
\usepackage{graphicx}


\title{Life-Cycle Assessment for Sustainable Aviation}


\author{
    \href{https://orcid.org/0000-0003-4859-2650}
    {\includegraphics[scale=0.06]{aux/figures/orcid.pdf}
    \hspace{1mm}
    Michael Weinold} \\
	Paul Scherrer Institut\\
	Laboratory for Energy Systems Analysis\\
	Group for Technology Assessment\\
	OSHA/D22\\
    5232 Villigen PSI \\
    Switzerland \\
	\texttt{michael.weinold@psi.ch} \\
}

\date{28. July 2022}

\renewcommand{\headeright}{Doctoral Research Plan}
\renewcommand{\undertitle}{Doctoral Research Plan}

%%% Add PDF metadata to help others organize their library
%%% Once the PDF is generated, you can check the metadata with
%%% $ pdfinfo template.pdf
\hypersetup{
pdftitle={A template for the arxiv style},
%pdfsubject={q-bio.NC, q-bio.QM},
pdfauthor={Michael Weinold},
%pdfkeywords={First keyword, Second keyword, More},
}

\begin{document}
\maketitle

\begin{abstract}
	\cite{becattini_role_2021}
\end{abstract}


\section{Introduction}

\cite{prussi_corsia_2021}

how are sustainable aviation fuels defined -> CORSIA \cite{prussi_corsia_2021}

    Carbon emissions estimations for individual flights have been available from major flight search providers, including \textit{Google Flights} \cite{holden_google_2021}, \textit{Kayak} \cite{noauthor_kayak_2021} and \textit{SkyScanner} \cite{crosthwaite_how_2021} from 2021. Early studies have shown that the prominent display of carbon emissions can have positive enironmental impact by nudging consumers to choose flights that XXX \cite{amenta_adding_2020}\cite{sanguinetti_nudging_2022}.
    
    However, different methodologies are used to estimate carbon emissions, leading to significant differences in total estimated carbon emissions. For instance, the development team of \textit{Google Flights} changed the aclculation methodology to account for direct-air carbon emissions only, disregarding non-carbon emissions \cite{ali_commit_2022}. This has drawn criticism of \textit{"airbrushing emissions"} by environmental non-governmental organizations \cite{hern_google_2022}\cite{rowlatt_google_2022}.

\section{Hybrid Life Cycle Assessment}

	What is life-cycle assessement?
	Why is its use in aviation a novelty?
	
	Some recent publications have completely disregarded the impact of life-cycle emissions under the assumption that these would be negligible \cite{brazzola_definitions_2022}.
	
	Why is the use of hybrid life-cycle inventory methods in aviation a novelty?
	
\section{USP}
	
    Existing knowledge within PSI, for instance the recent thesis by Saad \cite{saad_synthetic_2022} and unpublished work by \cite{sacchi_climate-neutral_2022}.

\section{Ideas}

    Connect this to "M2.1: Publication analysing impact of multiple standards and standards implementations in LCA"

    Connect this to "RECCE 2035: Resource to Climate Comparison Evaluator"

	Chris: research and development phase often missed completely in LCA
	Chris: Airbus is using Brightway in future aircraft design
	Chris: knowledge in the group on sustainable fuels (Tom Terlouw?)
	Laura: Cambridge Aviation Impact Accelerator meeting in September - maybe meet before that?

\section{Questions}

How do the results from Romain's preprint \cite{sacchi_climate-neutral_2022} compare against the new ICAO paper \cite{prussi_corsia_2021}

\bibliographystyle{IEEEtran}
\bibliography{references}

\end{document}
